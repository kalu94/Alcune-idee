\documentclass[a4paper]{book}
\usepackage[utf8]{inputenc}
\usepackage[Lenny]{fncychap}
\usepackage[a4paper, twoside]{geometry} 
\usepackage{graphicx}       
\usepackage{amsfonts}           
\usepackage{hyperref}
\usepackage{amsthm}          
\usepackage{cleveref}
\usepackage{frontespizio}
\usepackage{tikz}
\usepackage{tikz-cd}
\usepackage{amssymb}
\usepackage{array}
\usepackage{hhline}
\usepackage{enumitem}
\usepackage{fancyhdr}
\usepackage{amsmath} 
\usepackage[italian]{babel}
\hypersetup{ 
    colorlinks = true,
    linkcolor = {blue},
    citecolor={red}
}
 
 \theoremstyle{plain}
\newtheorem{thm}{Theorem}[section]
\newtheorem{lem}[thm]{Lemma}
\newtheorem{prop}[thm]{Proposizione}
\newtheorem{cor}[thm]{Corollario}
\theoremstyle{definizione}
\newtheorem{defn}[thm]{Definizione}
\theoremstyle{remark}
\newtheorem{rem}[thm]{Nota}
\numberwithin{equation}{section}

\DeclareMathOperator{\id}{id}
\DeclareMathOperator{\Stab}{Stab}
\DeclareMathOperator{\Gal}{Gal}
\DeclareMathOperator{\Imm}{Im}
\DeclareMathOperator{\Rea}{Re}
\DeclareMathOperator{\ord}{ord}
\DeclareMathOperator{\Ker}{Ker}
\DeclareMathOperator{\im}{Im}
\DeclareMathOperator{\Frob}{Frob}
\DeclareMathOperator{\car}{char}
\DeclareMathOperator{\disc}{disc}
\newcommand{\oo}{\mathcal{O}}
\newcommand{\NN}{\mathcal{N}}
\newcommand{\Z}{\mathbb{Z}} 
\newcommand{\Q}{\mathbb{Q}} 
\newcommand{\C}{\mathbb{C}}
\newcommand{\N}{\mathbb{N}}
%opening
\begin{document}
\chapter{Alcune idee}
\section{Sulla lipshitzianità di $\Psi$}
	Sia $\Psi$ l'operatore definito a pag 14.
	Ovvero, 
	$$\Psi\mu(t)=-\int_a\theta^a e^{\theta^at}E_0^a\bar{m}_0(da)+\frac{\alpha}{2k}\int_t^Tds\mu(s)\int_a e^{\theta^a(t-s)}\bar{m}_0(da)+$$
	$$-\frac{\alpha}{2k}\int_0^td\tau\int_\tau^Tds\mu(s)\int_a \theta^ae^{\theta^a(2\tau-t-s)}\bar{m}_0(da).$$
	
	\begin{defn}
	Chiamiamo $$\varphi(t)=\int_a e^{\theta^a(t-s)}\bar{m}_0(da).$$
		\end{defn}
		
	\begin{rem}
	Essendo la funzione generatrice dei momenti di una variabile aleatoria positiva, $\varphi(t)\le 1$ per $t\le 0$.
	\end{rem}
		
	\begin{prop}
	$$\Psi\mu_1(t)-\Psi\mu_2(t)=\frac{\alpha}{2k}\int_0^T ds (\mu_1(s)-\mu_2(s))G(s,t)$$
	dove
		$$G(t,s)=\begin{cases}
	-\ \displaystyle\frac{\varphi(s-t)-\varphi(-s-t)}{2} & s\le t, \\
	\displaystyle\frac{\varphi(t-s)+\varphi(-s-t)}{2} & s>t.
	\end{cases}$$
	\end{prop}
	\begin{proof}
		$$\Psi(\mu_1)(t)-\Psi(\mu_2)(t)=\frac{\alpha}{2k}\int_t^Tds (\mu_1(s)-\mu_2(s))\int_a e^{\theta^a(t-s)}\bar{m}_0(da)+$$
	$$-\frac{\alpha}{2k}\int_0^td\tau\int_\tau^Tds (\mu_1(s)-\mu_2(s))\int_a \theta^ae^{\theta^a(2\tau-t-s)}\bar{m}_0(da)=$$
	$$=\frac{\alpha}{2k}\int_t^Tds (\mu_1(s)-\mu_2(s))\varphi(t-s)+$$
	$$-\frac{\alpha}{2k}\int_0^td\tau\int_\tau^Tds (\mu_1(s)-\mu_2(s))\varphi'(2\tau-t-s)=$$
		$$=\frac{\alpha}{2k}\int_t^Tds (\mu_1(s)-\mu_2(s))\varphi(t-s)+$$
	$$-\frac{\alpha}{2k}\bigg(\int_0^tds (\mu_1(s)-\mu_2(s))\displaystyle\frac{\varphi(s-t)-\varphi(-s-t)}{2}+\int_t^Tds (\mu_1(s)-\mu_2(s))\displaystyle\frac{\varphi(t-s)-\varphi(-s-t)}{2}\bigg)=$$
	$$=\frac{\alpha}{2k}\int_0^T ds (\mu_1(s)-\mu_2(s))G(s,t)$$
	\end{proof}
	
	\begin{cor}
	$$\Vert\Psi(\mu_1)-\Psi(\mu_2)\Vert_\infty\le \frac{\alpha T}{2k}\Vert\mu_1-\mu_2\Vert_\infty,$$ tenendo conto della nota precedente.
	\end{cor}
	
	\section{Osservazione sul learning}
	
	Assumiamo che valga la disuguaglianza $$\frac{\alpha T}{2k}<1,$$ che, per quanto dimostrato prima, assicura l'esistenza e unicita' della distribuzione ottima $\mu(t)$.
	\\
	
	Supponiamo che tutti i giocatori ipotizzino una stima $\tilde\mu(t)$ della distribuzione ottima $\mu$, e che giochino quella. Al termine del gioco osservano una somma $m(t)$ delle velocità di liquidazione. 
	\\
	
	Siamo interessati a capire quanto fosse sbagliata la stima $\tilde\mu$, per aggiustare il tiro al prossimo gioco. Soprattutto, siamo interessati a sapere se fosse sbagliata in difetto o in eccesso. Quindi può essere interessante domandarsi: che relazione intercorre fra l'errore commesso $\tilde\mu(t)-\mu(t)$ e la differenza riscontrata $\tilde\mu(t)-m(t)$ (che può essere misurata)? L'articolo non ci dice nulla a tal riguardo. 
	
	\begin{prop}
	$$\int_0^Tdt (\tilde\mu(t)-\mu(t))F(t)=\int_0^Tdt (\tilde\mu(t)-m(t))(t)$$
	dove $F(t)$ e' una funzione $C^\infty$, decrescente, strettamente positiva, che vale $1$ in $0$.
	
	Piu' esattamente, abbiamo la seguente espressione esplicita (sorprendentemente semplice) per $F(t)$:
	$$F(t)=1-\frac{\alpha}{4k}\int_{-t}^tdr\varphi(r-T).$$
	
	\end{prop}
	\begin{proof}
	$$\tilde\mu(t)-\mu(t)=(\tilde\mu(t)-m(t))+(m(t)-\mu(t))=$$
	$$=(\tilde\mu(t)-m(t))+(\Psi\tilde\mu(t)-\Psi\mu(t))=$$
	$$=(\tilde\mu(t)-m(t))+\frac{\alpha}{2k}\int_0^T ds (\mu_1(s)-\mu_2(s))G(s,t),$$
	per la proposizione precedente. Integrando in $t$, otteniamo
	$$\int_0^T dt (\tilde\mu(t)-\mu(t))=\int_0^T dt(\tilde\mu(t)-m(t))+\frac{\alpha}{2k}\int_0^T ds (\tilde\mu(s)-\mu(s))\int_0^T dt G(s,t),$$ ovvero 
	$$\int_0^Tdt (\tilde\mu(t)-\mu(t)) \bigg(1-\frac{\alpha}{2k}\int_{0}^TdrG(t,r)\bigg)=\int_0^Tdt (\tilde\mu(t)-m(t))(t)$$
	Resta da dimostrare che 
	$$\int_{0}^TdrG(t,r)=\frac{1}{2}\int_{-t}^tdr\varphi(r-T).$$
	e che la funzione $F(t)$ definita sopra rispetta effettivamente le condizioni date, ma sono entrambe semplici verifiche. Per quanto riguarda la positività, basti osservare che 
	$$\frac{\alpha}{4k}\int_{-T}^Tdr\varphi(r-T)< \frac{\alpha T}{2k}<1$$ per ipotesi.
	\end{proof}
	
\end{document}